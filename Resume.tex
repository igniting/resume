% LaTeX CV

\documentclass[10pt]{article}
% Depends on:
\usepackage{ifthen}
\usepackage{geometry}
\usepackage[usenames,dvipsnames]{xcolor}
\usepackage{marvosym}
\usepackage{wasysym}
\usepackage{enumitem}
\usepackage{setspace}
\usepackage{xcolor}
\usepackage{tabularx}
\usepackage{multirow}

% Optionally remove page numbers:
%\thispagestyle{empty}\pagestyle{empty}

% No indent and increased paragraph spacing
\setlength{\parindent}{0pt}

% Configure margins:
\geometry{top=0.5cm,bottom=0.5cm,left=0.8cm,right=0.8cm}

% Commands
% - \cvtitle{Name}{ColorForName}{SpaceBelow} - Creates the CV title
\newcommand{\cvtitle}[2]{
	{\Huge\textsc{#1}}\\[#2]
}

% - \positionedbox{TextSide}{Width}{TextContent} - Creates a box of given width with text aligned as specified by 'TextSide' of given width. Note if you use box sizes which are less that the width of the page they can wrap together , ie 2 boxes of size 0.5\textwidth will sit side by side
\newcommand{\positionedbox}[3]{%
	\begin{minipage}{#2}
		\begin{flush#1}
			#3
		\end{flush#1}
	\end{minipage}}

% - \information{pieceofinfo}{pic/rightpic/space/none}{character} - Displays a piece of information pre/post-fixed with a 'character' symbol (pic), or leading space (space) or nothing (none).
% characters: phone = \Telefon, letter = \Letter, email = \MVAt, URL = \Flatsteel, twitter = $\mathcal{T}$
% NOTE: at the moment if you specify 'none' for the type, you still should put a non breaking space for 'character'
\newcommand{\information}[3]{
		\ifthenelse{\equal{#2}{pic}}
			{#3}
			{\ifthenelse{\equal{#2}{space}}
				{~ ~ }
				{}
			}
		#1\ifthenelse{\equal{#2}{rightpic}}
			{~ #3}
			{}
		}

% \cvheaderseperator{a}{b}{c}: Creates a horizontal rule separator where 'a' is the thickness, 'b' is the spacing above and 'c' is the spacing below.
\newcommand{\cvheaderseperator}[3]{\\[#2]\rule{\linewidth}{#1}\\[#3]}

% \cvskill{skill}{type}: Creates a 'skill' bullet. 'skill' is the actual skill text and 'type' is one of 'full', 'semi' or 'none' to indicate whether a full, half-full or empty bullet circle should be used. 
\newcommand{\cvskill}[2]{\hspace{4mm}%
	\ifthenelse{\equal{#2}{full}}%
		{\CIRCLE~ }%
		%else%
		{\ifthenelse{\equal{#2}{semi}}%
			{\LEFTcircle ~ }%
			{\Circle ~ }}#1\\}

% \cvsectiontitle{title}: Creates a section with the given title.
\newcommand{\cvsectiontitle}[1]{
    		\colorbox{gray!40}{%
        \begin{minipage}{0.989\linewidth}%
            \vspace*{1pt}%Space before
            \large\indent\textbf{#1}
            \vspace*{1pt}%Space after
        \end{minipage}%
   		}\\[1mm]
		}

% \cvpub{Authors}{Title}{Date}{Journal/Conference}{Pages}: Create a publication entry
\newcommand{\cvpub}[5]{#1. (#3) ``#2'' \emph{#4}, pages #5} 

% \cvcompany{Name}{Date}: A new company in the experience section
\newcommand{\cvcompany}[2]{\textbf{#1}\hspace{\stretch{3}}{(#2)}}

% \cvsublevel{Text}: Create an indented item list
\newcommand{\cvsublevel}[1]{\begin{itemize}[leftmargin=0.5cm] #1\end{itemize}}

% \cvsubbullet{Text}: Create a buller in the sub level
\newcommand{\cvsubbullet}[1]{\vspace{-1mm}\item #1}

% Modify List enviroment
\renewcommand{\labelitemi}{\RIGHTarrow}
\renewcommand{\labelitemii}{$\bullet$}
\renewcommand{\labelitemiii}{\RIGHTarrow}
\renewcommand{\labelitemiv}{\RIGHTarrow}

%%%%%%%%%%%%%%%%%%%%%%%%%%%%%%%%%%
%                                    Content                                       %
%%%%%%%%%%%%%%%%%%%%%%%%%%%%%%%%%%
\begin{document}
\cvtitle{Anshu Avinash}{0.1cm}
\information{+919005671853}{pic}{\Telefon}
\hfill
\information{anshuavi@iitk.ac.in, anshu.avinash35@gmail.com}{pic}{\MVAt}
%
\cvheaderseperator{0.1mm}{-0.1cm}{0.1mm}

% Contents Go Here
\vspace{-0.3cm}
\cvsectiontitle{EDUCATION}
\begin{tabularx}{\textwidth}{|llXl|}
\hline
\textbf{Year} 			& \textbf{Degree} 						& \textbf{Institution/Board} 						& \textbf{CPI/\%} \\ \hline
\multirow{2}{*}{2015 (expected)} & \multirow{2}{*}{BTech-MTech Dual Degree (CSE)} & \multirow{2}{*}{IIT Kanpur} & PG: 9.6/10.0 \\ & & & UG: 9.1/10.0 \\
2010 			& XII    						& KV ASC Center(S), Bangalore (CBSE)   	& 96.8\% \\
2008 			& X      						& KV ASC Center(S), Bangalore (CBSE)   	& 98.2\% \\ \hline
\end{tabularx}

% Skill set
\vspace{0.3cm}
\cvsectiontitle{TECHNICAL SKILLS}
\vspace{-0.6cm}
\begin{itemize}[leftmargin=0.5cm]
\setlength{\itemsep}{0.1mm}
\item Programming Languages: C/C++, Java, Python, Haskell, Scala, SQL, JavaScript
\item Frameworks and Libraries: OpenCV, OpenMPI, AngularJS and AngularDart, Play, Drupal, CUDA
\item Database: MySQL, Cassandra, Titan
\item Tools and packages: \LaTeX, bash scripting, git, make, octave
\end{itemize}

\cvsectiontitle{INTERNSHIPS}
\textbf{Self Tuning Optimizer in MariaDB} \hfill \textit{\textbf{Google Summer of Code}, May-August 2014} \\
\vspace{-0.6cm}
\begin{itemize}
\setlength{\itemsep}{0.1mm}
	\item Replaced the hard-coded constants in MySQL server optimizer with self tuning coefficients, resulting into better query plans
	\item Involved understanding MySQL server internals, coding in C++, version control using git, building a project using cmake
\end{itemize}
\textbf{Canary services for Facebook infrastructure} \hfill \textit{\textbf{Facebook Inc.}, Menlo Park, USA, May-July 2013} \\
\vspace{-0.6cm}
\begin{itemize}
\setlength{\itemsep}{0.1mm}
	\item Learnt about existing tools that various teams in Facebook use for deploying canary services
	\item Consolidated these tools into a more generic framework
	\item Involved working with large scale distributed systems that stand at foundation of Facebook infrastructure, advanced python coding, git, MySQL etc.
\end{itemize}

\cvsectiontitle{PROJECTS}
\textbf{Distributed web application for IIT Kanpur} \hfill \textit{Dec 2012-present}
\vspace{-0.2cm}
\begin{itemize}
\setlength{\itemsep}{0.1mm}
	\item Designed a web application to bring all the activities involving logistics and information organization at a central place
	\item Used Angular Dart for frontend development, Scala and Titan (Cassandra) database in backend, Jenkins for deployment
\end{itemize}
\textbf{Detection of malicious nodes in unstructured P2P} (\textit{Topics in Distributed Networks}) \hfill \textit{Aug-Nov 2013}
\vspace{-0.2cm}
\begin{itemize}
	\item Implemented Group Malicious Detection algorithm, similar to Too Late algorithm for detection of malicious peers in an unstructured P2P network using Java and PeerSim
\end{itemize}
\textbf{Higher Order Functions in C++} (\textit{Topics in Object Oriented Language Implementation}) \hfill \textit{Jan-April 2014}
\vspace{-0.2cm}
\begin{itemize}
	\item Implemented a type-safe framework of C++ class templates (called Functoid) for higher order functions
\end{itemize}
\textbf{Browser-X: A simple browser written in Haskell} (\textit{Functional Programming}) \hfill \textit{Jan-April 2013}
\vspace{-0.2cm}
\begin{itemize}
	\item Implemented a basic browser in Haskell with both GUI and CLI using Network.HTTP, gtk, webkit and sqlite3
\end{itemize}
\textbf{Other Course Projects}
\vspace{-0.2cm}
\begin{itemize}
\setlength{\itemsep}{0.1mm}
\item \textbf{Structure from Motion}: Estimated 3-D structures from 2-D image sequences coupled with local motion signals
\item \textbf{Compilers}: Created a compiler using LEX/YACC for the ADA language
\item \textbf{Computer Networks}: Implemented ipconfig utilty to display information about all network interfaces
\end{itemize}

\cvsectiontitle{SCHOLASTIC ACHIEVEMENTS}
\vspace{-0.5cm}
\begin{itemize}[leftmargin=0.5cm]
\setlength{\itemsep}{0.1mm}
\item Secured $2^{nd}$ position in Kendriya Vidyalaya Sanghatan (KVS) at national level in class X CBSE
\item Received Kishore Vaigyanik Protsahan Yogna (\textbf{KVPY}) scholarship in 2008
\item Received National Talent Search Exam (\textbf{NTSE}) scholarship in 2008
\item Attended International Mathematics Olympiad Training Camp (\textbf{IMO-TC}) organized at HBCSE, TIFR in 2009
\item Received \textbf{Academic Excellence Award} for year 2010-11 and 2011-12 at IIT Kanpur
\end{itemize}

\cvsectiontitle{POSITIONS OF RESPONSIBILITY}
\textbf{Coordinator, Programming Club} \hfill \textit{May 2012 - Mar 2013}
\vspace{-0.2cm}
\begin{itemize}
\item Conducted lectures and organized competitions on various topics including Competitive Programming, Web development and Application development
\end{itemize}
\textbf{Teaching Assistant for Algorithms II and ESC101}
\vspace{-0.2cm}
\begin{itemize}
\item Apart from designing and evaluating assignments, helped students in understanding the course content
\end{itemize}
\textbf{Link Student, Counselling Service} \hfill \textit{2012-13}
\vspace{-0.2cm}
\begin{itemize}
\item Helped an academically weak student in all of his courses
\end{itemize}
\end{document}
